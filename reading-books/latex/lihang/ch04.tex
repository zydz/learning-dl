% chap4.tex 
\ifx\allfiles\undefined
\documentclass{article}
\usepackage{xeCJK}
\begin{document}
\title{4 朴素贝叶斯法}
\date{}
\maketitle
\else
\chapter{第4章 朴素贝叶斯法}
\fi

\paragraph{}
朴素贝叶斯法是基于\textbf{贝叶斯定理}与\textbf{特征条件独立假设}的分类方法。


对于给定的训练数据集,首先基于特征条件独立假设学习输入、输出的联合概率分布;然后基于此模型,对给定的输入$x$,利用贝叶斯定理求出后验概率最大的输出$y$。


\section{4.1 朴素贝叶斯法的学习和分类}
\subsection{基本方法}
训练集:
$$
T = \{(x_1, y_1), (x_2, y_2), \cdots, (x_N, y_N) \}
$$
由$P(X,Y)$独立同分布产生。
先验概率分布:
\begin{equation}
P(Y=c_k), k=1,2,\cdots,K
\end{equation}
条件概率分布
\begin{equation}
P(X=x|Y=c_k)=P(X^{(1)}=x^{(1)},\cdots, X^{(n)}=x^{(n)}|Y=c_k), k=1,2,\cdots,K
\end{equation}

\section{4.2 。。。}
\ifx\allfiles\undefined
\end{document}
\fi